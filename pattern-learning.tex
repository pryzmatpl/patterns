\documentclass{article}

\usepackage{amsmath}
\usepackage{amssymb}  
\usepackage{amsthm}  
\usepackage{geometry} % For better margins
\usepackage{enumitem} % For improved lists
\usepackage{listings} % For code listing
\usepackage{hyperref} % For clickable links

% Define theorem-like environments
\newtheorem{theorem}{Theorem}[section]
\newtheorem{definition}[theorem]{Definition}
\newtheorem{lemma}[theorem]{Lemma}
\newtheorem{corollary}[theorem]{Corollary}
\newtheorem{proposition}[theorem]{Proposition}
\newtheorem{remark}[theorem]{Remark}

% Code listing style for assembly
\lstset{
    language=[x86masm]Assembler,
    basicstyle=\ttfamily\small,
    numbers=left,
    numberstyle=\tiny,
    stepnumber=1,
    numbersep=5pt,
    frame=single,
    breaklines=true,
    breakatwhitespace=true,
    tabsize=4
}

\begin{document}

\title{Mathematical Object Definition and Its Applications}
\author{}
\date{March 08, 2025}
\maketitle

\section{Introduction}
This document explores an abstract mathematical system inspired by symbolic dynamics and rewriting systems, with elements denoted by symbols such as $*$, $\mathbin{\&}$, $1$, and $x$. These elements interact through a set of rewriting rules and a length function, formalized below. The system is designed to model complex transformations with applications in computation and cryptography.

\section{Definition of the System}
The system operates over a set of symbols $\mathcal{S} = \{*, \mathbin{\&}, 1, x\}$ and is defined by the following rewriting rules:
\begin{align*}
    * &\rightarrow 1, \\
    \mathbin{\&} &\rightarrow *1, \\
    * + 1 &\rightarrow \mathbin{\&}*(1x), \\
    * + 1 &\rightarrow \mathbin{\&} + 1.
\end{align*}

\begin{remark}
    These relations are interpreted as non-deterministic rewriting rules rather than strict equalities.
\end{remark}

A length function $\text{Len}: \mathcal{S} \to \mathbb{R} \cup \{\infty\}$ is defined:
\begin{align*}
    \text{Len}(c) &= \begin{cases} 
    1, & \text{if } c = 1, \\
    x, & \text{if } c \in \{x, \mathbin{\&}\}, \\
    \infty, & \text{if } c = *.
    \end{cases}
\end{align*}

\section{Fixed Point Theorems in Symbol Substitution Spaces}
Consider a symbol space $\mathcal{S} = \{*, \mathbin{\&}, 1, x\}$ with a substitution mapping $\sigma: \mathcal{S} \to \mathcal{S}^*$:
\begin{align*}
    \sigma(*) &= 1 \text{ or } \mathbin{\&}1, \\
    \sigma(\mathbin{\&}) &= *1, \\
    \sigma(1) &= 1, \\
    \sigma(x) &= x.
\end{align*}

The fixed point set is:
\[
\mathcal{F}_\sigma = \{s \in \mathcal{S}^* : \exists n \in \mathbb{N}, \sigma^n(s) = s\}.
\]

\begin{theorem}
    The fixed point set $\mathcal{F}_\sigma$ contains at least one infinite sequence.
\end{theorem}

\begin{proof}
    Construct the sequence $\{s_n\}$ with $s_0 = *$ and $s_{n+1} = \sigma(s_n)$. Choosing $\sigma(*) = \mathbin{\&}1$ consistently:
    \[
    s_0 = *, \quad s_1 = \mathbin{\&}1, \quad s_2 = *11, \quad s_3 = \mathbin{\&}111, \quad \ldots
    \]
    This grows indefinitely, forming an infinite fixed point.
\end{proof}

\section{Cryptographic Applications}
\begin{definition}
    The extended length function $\hat{\text{Len}}: \mathcal{S}^* \to \mathbb{R} \cup \{\infty\}$ is:
    \[
    \hat{\text{Len}}(w) = \sum_{i=1}^{|w|} \text{Len}(w_i).
    \]
\end{definition}

\begin{definition}
    Define the hash function $H: \mathcal{S}^* \to \{0,1\}^k$ by:
    \[
    H(w) = \text{LSB}_k \left( \sum_{i=1}^{|w|} \hat{\text{Len}}(w_i) \cdot i \mod 2^k \right).
    \]
\end{definition}

\begin{theorem}
    $H$ is preimage-resistant under the assumption that computing $\hat{\text{Len}}(w)$ is computationally complex.
\end{theorem}

\begin{proof}
    Given $y = H(w)$, finding $w$ requires inverting $\hat{\text{Len}}(w)$, which may include $\infty$. The non-determinism in $\sigma$ suggests that enumerating preimages is intractable.
\end{proof}

\section{Assembly Implementation}
The following assembly code implements $H$, with \texttt{len\_function} handling symbol-specific lengths:

\begin{lstlisting}
section .text
hash_function:
    xor eax, eax        ; Clear accumulator
    xor edx, edx        ; Clear index counter
.loop:
    cmp edx, ecx        ; End of input check
    jge .done
    movzx ecx, byte [esi + edx] ; Load character
    call len_function   ; Compute Len
    imul ebx, ecx       ; Multiply by position (edx + 1)
    add eax, ebx        ; Add to hash
    inc edx             ; Next position
    jmp .loop
.done:
    and eax, 0xFFFFFFFF ; Modulo 2^32
    ret

len_function:
    cmp cl, '1'
    je .one
    cmp cl, '&'
    je .and
    cmp cl, '*'
    je .star
    mov ecx, 1          ; Default for x or others
    ret
.one:
    mov ecx, 1
    ret
.and:
    mov ecx, 2          ; Assume x = 2 for &
    ret
.star:
    mov ecx, 0xFFFFFFFF ; Represents infinity
    ret
\end{lstlisting}


\section{Conclusion}
This framework advances symbolic substitution with cryptographic and computational applications. Future work includes:
\begin{itemize}
    \item Formalizing rules as a context-sensitive grammar,
    \item Empirically testing $H$’s collision resistance,
    \item Exploring hardware acceleration.
\end{itemize}

\end{document}
